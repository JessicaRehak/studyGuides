\documentclass[letter]{article}

\usepackage{amsmath}
\usepackage{graphicx}
\usepackage{geometry}
\usepackage{braket} %Can do bra-ket notation with \braket{}
\usepackage{framed} %Adds the framed environment
\usepackage{fancyhdr}
\usepackage{datetime} %For formatting of header date
\usepackage{ulem} %Makes strike-through lines with \sout{}
\usepackage{booktabs} %better tables
\usepackage{multirow} %Support multi-row in tables
\usepackage[table,xcdraw]{xcolor} %Support colored rows in tables
\usdate %Month, Dth, YYYY
\geometry{
  letterpaper,
  left=1in,
  right=1in,
  bottom=1in,
  top=1in}
\pagestyle{fancy}
\lhead{NE120 Final Study Guide}
\chead{}
\rhead{}
\lfoot{}
\cfoot{\thepage}
\rfoot{\today \quad \currenttime}
\setlength\parindent{0pt}

\begin{document}
\textbf{\Large{Nuclear Engineering 120: Screener Study Guide}} \\
\vspace{12pt}
%\cite[pp. 45]{krane}
%\cite[Lec 24]{lecture}

\textbf{Disclaimer:} This is not an official study guide. Stuff \sout{might}
\textbf{is} wrong. Use the lecture notes and book!
\vspace{10pt}

\tableofcontents

\section{Nuclear Fuels}

\subsection{Types}
\begin{itemize}
\item Metallic: U and Pu-based. Advantages: high thermal conductivity,
  high fissile atom density, easy to fabricate. Issues: low melting
  points, unstable in irradiation, poor corrosion resistance.
\item Ceramic: UO$_2$, UC, UN, etc. Advantages: superior strength at
  high temperatures, low thermal expansion, good corrosion resistance,
  good stability under irradiation.
\end{itemize}

\subsection{Fission Products}
\begin{itemize}
\item Elemental gases: Kr, Xe, Cs, Rb, and Te
\item Metallic inclusions: Mo and noble metals
\item Oxide precipitates: Ba, Zr, and other
\item Dissolved oxides in the fuel matrix: Mo, Zr, rare earths
\end{itemize}

\end{document}