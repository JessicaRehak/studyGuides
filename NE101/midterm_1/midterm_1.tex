\documentclass[letter]{article}

\usepackage{amsmath}
\usepackage{graphicx}
\usepackage{geometry}
\usepackage{braket} %Can do bra-ket notation with \braket{}
\usepackage{framed} %Adds the framed environment
\usepackage{fancyhdr}
\usepackage{datetime} %For formatting of header date
\usepackage{ulem} %Makes strike-through lines with \sout{}
\usepackage{booktabs} %better tables
\usepackage{multirow} %Support multi-row in tables
\usepackage[table,xcdraw]{xcolor} %Support colored rows in tables
\usdate %Month, Dth, YYYY
\geometry{
  letterpaper,
  left=1in,
  right=1in,
  bottom=1in,
  top=1in}
\pagestyle{fancy}
\lhead{NE101 Midterm 1 Study Guide}
\chead{}
\rhead{}
\lfoot{}
\cfoot{\thepage}
\rfoot{\today \quad \currenttime}
\setlength\parindent{0pt}

\begin{document}
\textbf{\Large{Nuclear Engineering 101: Midterm 1 Study Guide}} \\
\vspace{12pt}
%\cite[pp. 45]{krane}
%\cite[Lec 24]{lecture}

\textbf{Disclaimer:} This is not an official study guide. Stuff \sout{might}
\textbf{is} wrong. Use the lecture notes and book!
\vspace{10pt}

\textbf{Note:} Everything in this guide is from the text (Krane) or
lecture, or office hours and should be cited as completely as
possible.

\tableofcontents

\section{Nuclear Properties}
\begin{itemize}
\item The mean nuclear radius is given by:
  \begin{equation*}
    R = R_0A^{1/3}
  \end{equation*}
where $A$ is the nuclear mass and $R_0$ is a constant (generally 1.2
fm).~\cite[pp. 48]{krane}
\item Nuclear binding energy is given by:
  \begin{equation*}
    B(Z,A) = [Zm(^1H) + Nm(n) - m(^A_ZX)]c^2
  \end{equation*}
where m$(^A_ZX)$ is the \textit{atomic} mass of the atom, and m($^1H$)
is the atomic mass of hydrogen. Make sure you use atomic, to ensure
the masses of the electrons cancel out.~\cite[Lec 2]{lecture}
\end{itemize}
\subsection{Semi-empirical Mass Formula}
\begin{itemize}
\item Factors that determine the amount of binding energy:
  \begin{itemize}
  \item The strong nuclear force is short range.
  \item Nucleons on the surface have less neighbors.
  \item Protons repel each other.
  \item Symmetry is important (Z $\approx$ N).
  \item Pairing is important (protons and neutrons like to pair up)
  \end{itemize}
These are all contained in the \textit{Bethe-Weizs\"acker
  Formula}.~\cite[Lec 3]{lecture}
\item The Semi-empirical mass formula (Bethe-Weisz\"acker Formula) is
  called that because it is based on physics, but uses empirical
  measurements to get the constants. Therefore, it is not derived from
  first principles, but is just a model of what we have
  observed.~\cite[Lec 3]{krane} This means it's \textit{wrong} and
  just represents our best estimate of what is happening.
\item The semi-empirical mass formula:
  \begin{equation*}
    \begin{split}
      B &= a_VA-
      a_SA^{2/3}-a_C\frac{Z(Z-1)}{A^{1/3}}-a_{\text{sym}}\frac{{(A-2Z)}^2}{A}
      + \delta{}(A,Z) \\
      \text{Binding Energy} &=  \text{Volume term} -  \text{surface
        term} -  \text{Coulomb Term} -  \text{Symmetry term} +
      \text{Pairing term}
    \end{split}
  \end{equation*}
Where $a_n$ are constants adjusted to make the calculated binding
energy match experimental masses. The pairing term:
\begin{equation*}
  \delta = 
  \begin{cases}
    +a_PA^{-3/4} & \text{for Z and N even} \\
    -a_PA^{-3/4} & \text{for Z and N odd} \\
    0 & \text{for A odd}
  \end{cases}
\end{equation*}


\end{itemize}

\section{Radioactive Decay}
\begin{itemize}
\item The decay constant $\lambda$ (units sec$^{-1}$) is the probability
  per unit time that an atom will decay. The number of atoms decaying
  per time is given by:
  \begin{equation*}
    \frac{dN}{dt} = -\lambda{}N
  \end{equation*}
Where $N$ is the total number of radioactive nuclei
present. This can be solved to get:
\begin{equation*}
  N(t) = N_0e^{-\lambda{}t}
\end{equation*}
Where $N_0$ is the initial amount of nuclei presents when
$t=0$.~\cite[pp. 161]{krane}
\item The half-life is the time for half the initial number of nuclei
  to decay:
  \begin{equation*}
    t_{1/2} = \frac{0.693}{\lambda}
  \end{equation*}
The \textit{mean lifetime} $\tau$ is a bit different and is average
amount of time before a radioactive nuclei will decay:
\begin{equation*}
  \tau = \frac{1}{\lambda}
\end{equation*}
I don't know when you'd actually use this.~\cite[pp. 161]{krane}
\end{itemize}


\bibliographystyle{unsrt}
\bibliography{../NE101}
\end{document}
