\documentclass[letter]{article}

\usepackage{amsmath}
\usepackage{graphicx}
\usepackage{geometry}
\usepackage{braket} %Can do bra-ket notation with \braket{}
\usepackage{framed} %Adds the framed environment
\usepackage{fancyhdr}
\usepackage{datetime} %For formatting of header date
\usepackage{ulem} %Makes strike-through lines with \sout{}
\usepackage{booktabs} %better tables
\usepackage{multirow} %Support multi-row in tables
\usepackage[table,xcdraw]{xcolor} %Support colored rows in tables
\usdate %Month, Dth, YYYY
\geometry{
  letterpaper,
  left=1in,
  right=1in,
  bottom=1in,
  top=1in}
\pagestyle{fancy}
\lhead{NE150 Midterm 1 Study Guide}
\chead{}
\rhead{}
\lfoot{}
\cfoot{\thepage}
\rfoot{\today \quad \currenttime}
\setlength\parindent{0pt}

\begin{document}
\textbf{\Large{Nuclear Engineering 150: Midterm 1 Study Guide}} \\
\vspace{12pt}
%\cite[pp. 45]{krane}
%\cite[Lec 24]{lecture}

\textbf{Disclaimer:} This is not an official study guide. Stuff \sout{might}
\textbf{is} wrong. Use the lecture notes and book!
\vspace{10pt}

\textbf{Note:} Everything in this guide is from the text () or
lecture, or office hours and should be cited as completely as
possible.

\tableofcontents

\section{Fundamentals}
\label{sec:fundamentals}

\subsection{Atomic Masses}
\begin{enumerate}
\item Atomic mass unit (amu): defined by the mass of a neutral
  carbon-12 atom.
  \begin{equation*}
    m(^{12}\mathrm{C}) = 12 \mathrm{ amu}
  \end{equation*}
\item A mixture of an element with various isotopes $i$ with
  abundances $\gamma_i$ is:
  \begin{equation*}
    M = \sum_i\gamma_i{}M_i
  \end{equation*}
You can use this for the natural abundances, or if you just have a
mixture of elements in a compound (like enriched uranium).
\end{enumerate}

\bibliographystyle{unsrt}
\bibliography{../NE150}
\end{document}