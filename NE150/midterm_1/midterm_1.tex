\documentclass[letter]{article}

\usepackage{amsmath}
\usepackage{graphicx}
\usepackage{geometry}
\usepackage{braket} %Can do bra-ket notation with \braket{}
\usepackage{framed} %Adds the framed environment
\usepackage{fancyhdr}
\usepackage{datetime} %For formatting of header date
\usepackage{ulem} %Makes strike-through lines with \sout{}
\usepackage{booktabs} %better tables
\usepackage{multirow} %Support multi-row in tables
\usepackage[table,xcdraw]{xcolor} %Support colored rows in tables
\usdate %Month, Dth, YYYY
\geometry{
  letterpaper,
  left=1in,
  right=1in,
  bottom=1in,
  top=1in}
\pagestyle{fancy}
\lhead{NE150 Midterm 1 Study Guide}
\chead{}
\rhead{}
\lfoot{}
\cfoot{\thepage}
\rfoot{\today \quad \currenttime}
\setlength\parindent{0pt}

\begin{document}
\textbf{\Large{Nuclear Engineering 150: Midterm 1 Study Guide}} \\
\vspace{12pt}
%\cite[pp. 45]{krane}
%\cite[Lec 24]{lecture}

\textbf{Disclaimer:} This is not an official study guide. Stuff \sout{might}
\textbf{is} wrong. Use the lecture notes and book!
\vspace{10pt}

\textbf{Note:} Everything in this guide is from the text () or
lecture, or office hours and should be cited as completely as
possible.

\tableofcontents

\section{Fundamentals}
\label{sec:fundamentals}

\subsection{Masses}
\begin{itemize}
\item Atomic mass unit (amu): defined by the mass of a neutral
  carbon-12 atom.
  \begin{equation*}
    m(^{12}\mathrm{C}) = 12 \mathrm{ amu}
  \end{equation*}
\item A mixture of an element with various isotopes $i$ with
  abundances $\gamma_i$ is:
  \begin{equation*}
    M = \sum_i\gamma_i{}M_i
  \end{equation*}
You can use this for the natural abundances, or if you just have a
mixture of elements in a compound (like enriched uranium).~\cite[Lec
3]{lecture}
\item To get molecular mass, you can just add the molar masses of the
  component atoms (multiplied by how many of each atom you have in the
  molecule).
\end{itemize}

\subsection{Nuclear and Atomic Radii}
\begin{itemize}
\item The average atomic radii is the same for all atoms, about
  $2\times{}10^{-10}$ m.~\cite[Lec 3]{lecture}
\item The nuclear radii is approximated by:
  \begin{equation*}
    r=r_0A^{1/3}
  \end{equation*}
Where $r_0 \approx 1.25$ fm.~\cite[Lec 3]{lecture}
\end{itemize}

\subsection{Number (Atom) Density}

\begin{itemize}
\item The mass density (usually in g/cm$^3$) is given by:
  \begin{equation*}
    \rho(^AX) = \frac{N(^AX)M(^AX)}{N_A}
  \end{equation*}
Where $N(^AX)$ is the atom density (atoms/cm$^3$), $M(^AX)$ is the
molar mass, and $N_A$ is avogadro's number.~\cite[Lec 3.]{lecture} This might be useful but
the mass density is usually one of the given things in problems.
\item The atom density (atoms/cm$^3$) is given by:
  \begin{equation*}
    N(^AX)=\frac{\rho(^AX)N_A}{M(^AX)}
  \end{equation*}
Where all the terms were defined in the previous part.~\cite[Lec
3]{lecture}
\item For a molecule of form $X_mY_n$, the atom density of the
  components $X$ and $Y$ are:
  \begin{equation*}
    \begin{split}
      N_X &= mN_{X_mY_n} \\
      N_Y &= nN_{X_mY_n}
    \end{split}
  \end{equation*}
I think that's kind of obvious but it's there ok.~\cite[Lec
3]{lecture}
\end{itemize}

\subsection{Radioactive Decay}

\begin{itemize}
\item The rate at which an isotope decays is given by:

\end{itemize}

\section{Reactions}

\subsection{Cross-sections}

\begin{itemize}
\item Cross-sections are characteristic of the probability of a
  reaction occurring. They are \textbf{not} probabilities, they are
  \textbf{proportional} to probabilities.
\end{itemize}


\bibliographystyle{unsrt}
\bibliography{../NE150}
\end{document}