\documentclass[letter]{article}

\usepackage{amsmath}
\usepackage{graphicx}
\usepackage{geometry}
\usepackage{braket} %Can do bra-ket notation with \braket{}
\usepackage{framed} %Adds the framed environment
\usepackage{fancyhdr}
\usepackage{datetime} %For formatting of header date
\usepackage{ulem} %Makes strike-through lines with \sout{}
\usepackage{booktabs} %better tables
\usepackage{multirow} %Support multi-row in tables
\usepackage[table,xcdraw]{xcolor} %Support colored rows in tables
\usdate %Month, Dth, YYYY
\geometry{
  letterpaper,
  left=1in,
  right=1in,
  bottom=1in,
  top=1in}
\pagestyle{fancy}
\lhead{NE150 Midterm 1 Study Guide}
\chead{}
\rhead{}
\lfoot{}
\cfoot{\thepage}
\rfoot{\today \quad \currenttime}
\setlength\parindent{0pt}

\begin{document}
\textbf{\Large{Nuclear Engineering 150: Midterm 2 Study Guide}} \\
\vspace{12pt}
%\cite[pp. 45]{krane}
%\cite[Lec 24]{lecture}

\textbf{Disclaimer:} This is not an official study guide. Stuff \sout{might}
\textbf{is} wrong. Use the lecture notes and book!
\vspace{10pt}

\textbf{Note:} Everything in this guide is from the text () or
lecture, or office hours and should be cited as completely as
possible.

\tableofcontents

\section{Neutron Slowing Down}

\subsection{Lethargy}

\begin{itemize}
\item Lethargy ($u$) is a measure of the amount that a neutron has slowed
  down \textit{relative} to an energy $E_0$. It is important to note
  that this is a relative measure, it only tells us about the
  neutron's energy when compared to our reference value.\cite[Lec. 9]{lecture}
  \begin{equation*}
    u = \ln\left(\frac{E_0}{E}\right)=\ln(E_0)-\ln(E)
  \end{equation*}
\item As the neutron slows down, relative to $E_0$, the value of
  lethargy goes \textit{up}. This is why we call it a measure of the
  slowing down. This is why we call it lethargy because it's like a
  measure of the neutron's sleepiness or something; some nuclear
  engineer obviously thought he was being super cute.\cite[Lec. 9]{lecture}
\item Every collision causes a decrease in neutron energy (and
  increase in neutron lethargy). If the neutron goes from $E_i$ to
  $E_f$ after a scattering event, we can solve for the difference in
  the initial and final energies.
  \begin{equation*}
    \Delta{}u=\ln\left(\frac{E_0}{E_f}\right)-\ln\left(\frac{E_0}{E_i}\right)=\ln\left(\frac{E_i}{E_f}\right)
  \end{equation*}
This is also called the logarithmic energy loss.\cite[Lec.9]{lecture}
\end{itemize}

\subsubsection{Average Logarithmic Energy Loss per collision}

\begin{itemize}

\item If we want to know the average logarithmic energy loss per
  collision, we can take the average. We call that squiggle (written
  $\xi$). We will use the energy loss per collision from the last
  midterm, and denote the original energy $E$ and the final energy
  $E'$:
  \begin{equation*}
    \xi = \overline{\ln\left(\frac{E}{E'}\right)} =
    \int^E_{\alpha{}E}dE'\ln\left(\frac{E}{E'}\right)p(E\to{}E') = \int^E_{\alpha{}E}dE'\ln\left(\frac{E}{E'}\right)\frac{1}{(1-\alpha)E}
  \end{equation*}
Finally, you get:
\begin{equation*}
  \xi = 1+\frac{\alpha}{1-\alpha}\ln(\alpha)
\end{equation*}
Note that this doesn't depend on the original energy $E$. This makes
sense, because it's just an average value, we integrated over all the
possible energies. In the end, it's just a function of what it's
colliding with, because:
\begin{equation*}
  \alpha = \left(\frac{A-1}{A+1}\right)^2
\end{equation*}
\cite[Lec. 9]{lecture}
\item Note that for hydrogen ($A=1$), the value of $\alpha$ is 0. This
  makes $\xi$ undefined ($\ln(0)$ is undefined), so we just set $\xi$
  for hydrogen at 1 (there might be a mathematical way to justify this
  but who cares). This is the highest $\xi$ can be:
  \begin{equation*}
    \xi = 1 = \ln\left(\frac{E}{E'}\right) \to E'=\frac{E}{e}
  \end{equation*}
\item You might be saying ``But study guide, can't the neutron lose
  \textit{all} of its energy in a collision with a proton? Like
  billiard balls?'' and I'd say ``who says billiard balls, they're
  pool balls, get it together.'' Remember this is the \textit{average} energy lost per
  collision so its between the maximum lost (all of it) and the
  minimum lost (none of it).
\item The value
  of $\xi$ gets smaller and smaller with increasing $A$; bouncing a
  ping-pong ball off a basketball won't do much to slow the ping-pong
  ball down.
  \begin{equation*}
    \xi \approx \frac{2}{A+2/3}, \text{for } A>10
  \end{equation*}
Smaller $A \to$ larger $\xi$ (more effective at slowing). \cite[Lec. 9]{lecture}
\item We can use this to figure out the average number of elastic collisions
  required to go from an energy $E_1$ to $E_2$ (Note that we use
  higher numbers for lower energies, this is pretty standard notation
  in neutronics; this is why $E_0$ was our \textit{highest} relative
  energy when defining lethargy).
  \begin{equation*}
    n=\frac{\Delta{}u}{\xi}=\frac{\ln\left(\frac{E_1}{E_2}\right)}{\xi}
  \end{equation*}
  This is just the total logarithmic difference of our two energies,
  divided by the logarithmic amount of energy lost in each collision
  (distance divided by rate).\cite[Lec. 9]{lecture}
\end{itemize}

\subsubsection{$\xi$ for Molecules}
\begin{itemize}
\item If you have a molecule, you sum over the individual values of
  $\xi$, weighted by their cross-sections. This is to take into account
  that there's a different probability that the neutron will scatter
  off of the different elements in the molecule.
  \begin{equation*}
    \bar{\xi}=\frac{1}{\Sigma_s}\sum_i\xi_i\Sigma_{si}
  \end{equation*}
Where $\Sigma$ is the total scattering cross-section for the molecule
and $\Sigma_{Si}$ is the scattering cross-section for each element
$i$.\cite[Lec. 9]{lecture}
\item Remember to get the total scattering cross-section for the
  molecule, you just sum based on how many of each you have. For
  example, for water:
  \begin{equation*}
    \Sigma^{H_2O}_s=N_{H_2O}(2\sigma^H+\sigma^O)
  \end{equation*}
The number densities should all cancel out, letting you just use
microscopic cross-sections.
\end{itemize}

\section{Reactor Criticality}

\subsection{Neutron Population}
\begin{itemize}
\item We generally view the neutron population in our reactor in
  ``generations.'' This is an artificial construct, but is a good way
  to understand the inner mechanics of a reactor. We model our fission
  chain reaction as generating a whole bunch of neutrons at once (a
  generation) that then go on and live their lives. Ultimately, the
  neutrons will either leak, be absorbed, go on to cause more
  fission, etc. The fissions will then create the next generation of
  neutrons.
\item Leakage and absorption are examples of loss mechanisms, and
  fission is a production mechanism. Overall, if we could take the
  whole loss rate $L(t)$ and the whole production rate $P(t)$, we
  could calculate the change in neutrons between generations:
  \begin{equation*}
    \frac{dn(t)}{dt}=P(t)-L(t)
  \end{equation*}
\end{itemize}

\subsubsection{Multiplication Factor}

\begin{itemize}
\item The multiplication factor is the ratio of neutrons in the
  current generation, to those in the last
  generation.\cite[Lec. 10]{lecture}
  \begin{equation*}
    k\equiv\frac{\text{Number of neutrons in this generation}}{\text{Number
        of neutrons in the last generation}}
  \end{equation*}
\item There are three possible situations\cite[Lec. 10]{lecture}:
  \begin{itemize}
  \item[] $k < 1$: Subcritical; there are less neutrons in this
    generation than last, population is decreasing
  \item[] $k=1$: Critical: there are the same number of neutrons in
    this generation as the last, the neutron population is steady.
  \item[] $k>1$: Supercritical: there are more neutrons in this
    generation than the last, population is increasing.
  \end{itemize}
\item Alternatively, we can also define $k$ using the production rate
  of neutrons and the loss rate:
  \begin{equation*}
    k\equiv\frac{P(t)}{L(t)}
  \end{equation*}
You can see how this has the same three situations discussed
above.\cite[Lec. 10]{lecture}
\end{itemize}
\subsubsection{Neutron Population Lifetime}
\begin{itemize}
\item Using this loss rate $L(t)$, we can figure out how long our
  generation would survive (with no production). At a given time $t$,
  we have $n(t)$ neutrons, so using number over loss rate:
  \begin{equation*}
    \ell \equiv \frac{n(t)}{L(t)}
  \end{equation*}
This is our neutron generation lifetime.\cite[Lec 10.]{lecture}
\item We can combine this with our value of $k$ above by examining the
  change in our neutron population over time using the equation from above:
  \begin{equation*}
    \frac{dn(t)}{dt}=P(t)-L(t)=\left[\frac{P(t)}{L(t)}-1\right]L(t)=[k-1]\frac{L(t)}{n(t)}n(t)=\frac{k-1}{\ell}n(t)
  \end{equation*}
We can solve this using an initial condition $n(0)=n_0$:
\begin{equation*}
  n(t)=n_0\text{Exp}\left(\frac{k-1}{\ell}t\right)
\end{equation*}
\end{itemize}

\subsection{Four and Six Factor Formulas}
The four and six factors are values that multiply the number of
neutrons in the last generation ($n$) to get the number of neutrons in the
next generation. For this, we only assume that there are two
populations of neutrons, fast and thermal.

\subsubsection{Those Six Factors}

Start with $n$ fast neutrons created from fission.

\paragraph{Fast non-leakage probability}
There is a chance these fast neutrons will leak out of the
  reactor. The chance that this \textit{doesn't} happen is
  $P_{\text{FNL}}$, or the \textbf{fast non-leakage probability}. This
  means that $nP_{\text{FNL}}$ fast neutrons are left, and
  $n(1-P_{\text{FNL}})$ leak out of the reactor.
\paragraph{Resonance escape probability}
The fast neutrons that don't leak will collide with the
  moderator and slow down. While slowing down, they must pass through
  the resonance region where there is an enhanced chance of capture in
  the fuel. The probably of \textit{not} being absorbed while slowing
  down to thermal energies is $p$, the \textbf{resonance escape
    probability}. Therefore, $npP_{\text{FNL}}$ fast neutrons reach
  thermal energies.
  \begin{framed}
    Resonance escape probability is strongly affected by moderator/fuel ratio
    and temperature. As temperature rises and the moderator gets less
    dense (or we have less moderator somehow), neutrons moderate
    slower, spending more time in the resonance region. Therefore,
    \textbf{p will decrease} because there is a higher probability
    that they'll be absorbed. Also, as fuel temperature increases,
    \textbf{p will decrease} because Doppler broadening increases the
    probability of capture.\cite[Lec. 10]{lecture}
  \end{framed}

\paragraph{Thermal non-leakage probability}
These thermal neutrons can also escape, just like fast
  neutrons. Again, the chance this \textit{doesn't} happen is the
  \textbf{thermal non-leakage probability}, $P_{\text{TNL}}$. This
  gives us $npP_{\text{FNL}}P_{\text{TNL}}$ thermal neutrons that
  don't leak out.

\paragraph{Thermal utilization factor}
If a thermal neutron doesn't leak out, it has to go somewhere:
  it gets absorbed. But, just because the thermal neutron is absorbed,
  doesn't mean that it's absorbed in the fuel, or
  \textit{utilized}. It could be absorbed anywhere else in the
  reactor. This gives rise to the \textbf{thermal utilization factor}:
  \begin{equation*}
    f=\frac{\Sigma_a^{\text{fuel}}}{\Sigma^{\text{fuel}}_a+\Sigma^{\text{non-fuel}}_a}=\frac{\propto
      \text{Probability of absorption in fuel}}{\propto
      \text{Probability of absorption in anything}}
  \end{equation*}
Remember that cross-sections are characteristic of (proportional to) probabilities, so
dividing cross sections gives us an actual probability. All $f$ is is
the probability that given a neutron is absorbed, it is absorbed in
fuel. We knew that our thermal neutrons were absorbed, so we can just
tack this onto the end of our growing expression to get the number of
thermal neutrons absorbed in the fuel material:
$fnpP_{\text{FNL}}P_{\text{TNL}}$.

\begin{framed}
\textbf{Important Note:} The above formulation assumes you have a
homogeneous reactor. That is, the flux in the fuel and non-fuel
materials are the same, and the volume of the fuel is the same as the
volume of the non-fuel materials. If this isn't the case (heterogeneous), you have to
do more. For example, if we have only fuel and moderator of different
volumes and with different flux:
\begin{equation*}
  f=\frac{\Sigma^{\text{fuel}}_aV^{\text{fuel}}\phi^{\text{fuel}}}{\Sigma^{\text{fuel}}_aV^{\text{fuel}}\phi^{\text{fuel}}+\Sigma^{\text{mod}}_aV^{\text{mod}}\phi^{\text{mod}}}
\end{equation*}
Or, we can divide through by $V^{\text{fuel}}\phi^{\text{fuel}}$ to
get:
\begin{equation*}
  f=\frac{\Sigma^{\text{fuel}}}{\Sigma^{\text{fuel}}+\Sigma^{\text{mod}}_a\frac{V^{\text{mod}}}{V^{\text{fuel}}}\frac{\phi^{\text{mod}}}{\phi^{\text{fuel}}}}
\end{equation*}
The important thing to note here is that if
$V^{\text{fuel}}>V^{\text{mod}}$ then the fraction is $>1$ and $f$ can
be higher than for the homogeneous case.
\end{framed}

\textbf{Reproduction Factor}
Finally, we the number of neutrons that are actually absorbed in fuel
and we've almost come full circle. Now we just need to figure out how
many neutrons we get out of those absorptions to start the next
generation. For this, we use the \textbf{reproduction factor}:
\begin{equation*}
  \eta =\frac{\nu\Sigma_f}{\Sigma_f+\Sigma_\gamma}=\frac{\nu\Sigma_f}{\Sigma_a}
\end{equation*}
These are all cross-sections for the fuel. This isn't exactly equal to the average number of neutrons from
fission ($\nu$) because sometimes an absorption event doesn't result
in fission.  $\eta$ depends on
the fuel, and ranges from 1.34 for natural uranium to 2.08 for pure
uranium 235. Now, we have the number of fission neutrons due to
thermal fission: $\eta{}fnpP_{\text{FNL}}P_{\text{TNL}}$.
\begin{framed}
The difference between $\eta$ and $\nu$ can be confusing:
\begin{itemize}
\item $\nu$: The number of fast neutrons per \textit{fission}.
\item $\eta$: The number of fast neutrons per \textit{thermal neutron
    absorbed} by fuel. This will always be lower than $\nu$ because
  some amount of neutrons will be absorbed and not cause fission
  ($\Sigma_a$ above).
\end{itemize}
\end{framed}

\paragraph{Fast Fission Factor}

Some of the fast neutrons from the beginning might have caused fission
before thermalizing. To account for this, we use a \textbf{fast
  fission factor}. This is essentially a correction to what we ended
up with after the last step (the number of fission neutrons due to
thermal fission):
\begin{equation*}
  \epsilon = \frac{\text{Number of fission neutrons due to fast and
      thermal fission}}{\text{Number of fission neutrons due to
      thermal fission}}
\end{equation*}
So in the end, we have the total number of neutrons due to both fast
and thermal fission from our original population of neutrons $n$:
$\epsilon\eta{}fnpP_{\text{FNL}}P_{\text{TNL}}$. The value of
$\epsilon$ ranges from 1.00 to 1.10.
\begin{framed}
  The fast fission factor is affected by the moderator temperature and
  moderator to fuel ratio. As
  the moderator temperature rises and the moderator gets less dense,
  neutrons moderate slower, spending more time fast. This enhances the
  probability that they'll interact with fuel and create fast
  fission, so \textbf{$\epsilon$ will increase.}\cite[Lec. 13]{lecture}
\end{framed}

\subsubsection{The Six-Factor Formula}
Now, we can use this with our definition of the multiplication factor:
\begin{equation*}
  \begin{split}
    k&\equiv\frac{\text{Number of neutrons in this
        generation}}{\text{Number of neutrons in the last generation}}
    \\
    k&=\frac{\epsilon\eta{}fnpP_{\text{FNL}}P_{\text{TNL}}}{n}\\
    k&=\epsilon\eta{}fpP_{\text{FNL}}P_{\text{TNL}}
  \end{split}\end{equation*}
This is our six-factor formula. This allows us to multiply these
factors together and determine if the reactor is critical ($k=1$),
subcritical ($k<0$) or supercritical ($k>0$).

\subsubsection{Infinite Reactor ($k_\infty$)}

If we have an infinite reactor, there is no leakage:
$P_{\text{FNL}}=P_{\text{TNL}}=1$. Therefore, our equation reduces
down to four factors. We call this $k_\infty$ because it's for an
infinite reactor.\cite[Lec. 10]{lecture}
\begin{equation*}
  k_\infty=\epsilon{}pf\eta
\end{equation*}
Leakage is impossible to avoid unless your reactor is infinite, so
$k_\infty$ represents the upper bound of $k$.\cite[Lec. 10]{lecture}

\vspace{10pt}
We can also represent $k_\infty$ as a straight-up ratio of the
production rate of neutrons to the absorption rate (our only loss
rate)\cite[Lec. 13]{lecture}
\begin{equation*}
  k_\infty=\frac{\text{Production rate of neutrons}}{\text{Absorption
      rate of neutrons}}=\frac{\nu{}R_f}{R_a}=\frac{\nu\Sigma_f\Phi}{\Sigma_a\Phi}
\end{equation*}
I assume that the numerator includes both fast and thermal fission (as
$k_\infty$ has $\epsilon$ in it) so this formula is somewhat
confusing. Then I guess the flux in the denominator has to be broken
up as a sum of thermal and fast times their own cross-sections? In
reality, I think this equation just plays loosey-goosey with fast
fission and kind of ignores it (as it's a small factor). Just make
sure you understand what it's saying and why it's weird.

\subsubsection{Infinite Homogenous Reactor}
An infinite \textit{homogenous} reactor we still have no leakage
because the reactor is infinite. The homogenous part means that our
reactor is a completely uniform mixture of fuel and
moderator.\cite[Lec. 10]{lecture}  For
whatever reason, we assume this means that all neutrons are thermal
neutrons. I like to think of it as the \textit{perfect} mixture for
thermalizing. If a neutron isn't thermal, it will always then scatter
off a moderator and slow down more. If a neutron \textit{is} thermal,
it will always find a fuel nuclide to interact with. What this means
is that we don't have any fast fission (all fast neutrons always
scatter) and there is no resonance absorption (for the same
reason). This means our fast fission factor ($\epsilon$) and resonance
escape probability ($p$) are both 1.\cite[Lec. 10, 13]{lecture}
\begin{equation*}
  k_\infty=\epsilon{}pf\eta=f\eta=\frac{\Sigma^{\text{fuel}}_a}{\Sigma^{\text{fuel}}_a+\Sigma^{\text{non-fuel}}_a}\frac{\nu\Sigma_f^{\text{fuel}}}{\Sigma_a^{\text{fuel}}}=\frac{\nu\Sigma^{\text{fuel}}_f}{\Sigma^{\text{fuel}}_a+\Sigma^{\text{non-fuel}}_a}
\end{equation*}
\subsection{Energy Dependence}

Everything up until this point has assumed that we have only two
neutron energies (fast and thermal). If we want to treat these with
actual energy, we need to reintroduce the idea of flux weighted cross
sections. This is just averaging over the neutron energy
spectrum.\cite[Lec. 13]{lecture}
\begin{equation*}
  \overline{\Sigma}_x=\frac{\int^\infty_0\Sigma_x(E)\Phi(E)dE}{\int^\infty_0\Phi(E)dE}
\end{equation*}
We can use these in that equation for $k_\infty$ that I don't really
like.\cite[Lec. 13]{lecture}
\begin{equation*}
  k_\infty=\frac{\nu{}R_f}{R_a}=\frac{\nu{}\overline{\Sigma}_f}{\overline{\Sigma}_a}
\end{equation*}

\subsubsection{Thermal Disadvantage Factor}

Armed with average cross sections, we can go back to the thermal
utilization factor ($f$) for heterogeneous reactors without energy
dependence (from the framed section):
\begin{equation*}
    f=\frac{\Sigma^{\text{fuel}}}{\Sigma^{\text{fuel}}+\Sigma^{\text{mod}}_a\frac{V^{\text{mod}}}{V^{\text{fuel}}}\frac{\phi^{\text{mod}}}{\phi^{\text{fuel}}}}
\end{equation*}
We can now express these all as energy averaged cross-sections and
fluxes by very carefully drawing lines over the sigmas and replacing
$\phi$ with $\Phi$ (thank god for find and replace):
\begin{equation*}
    f=\frac{\overline{\Sigma}^{\text{fuel}}}{\overline{\Sigma}^{\text{fuel}}+\overline{\Sigma}^{\text{mod}}_a\frac{V^{\text{mod}}}{V^{\text{fuel}}}\frac{\overline{\Phi}^{\text{mod}}}{\overline{\Phi}^{\text{fuel}}}}
\end{equation*}
Remember that this is the \textit{thermal} utilization factor. All the
neutrons in these absorptions have been thermalized and didn't leak
out, so the fluxes shown are averaged over the thermal energies. We
define the \textbf{thermal disadvantage factor $\zeta$} as that flux
term in the denominator:
\begin{equation*}
  \zeta \equiv \frac{\overline{\Phi}^{\text{mod}}_{\text{th}}}{\overline{\Phi}^{\text{fuel}}_{\text{th}}}
\end{equation*}
Where ``th'' has been added to remind us that these are average
\textit{thermal} flux.\cite[Lec. 13]{lecture}

\subsection{Conversion and Breeding}



\bibliographystyle{unsrt}
\bibliography{../NE150}
\end{document} 