\documentclass[letter]{article}

\usepackage{amsmath}
\usepackage{graphicx}
\usepackage{geometry}
\usepackage{braket} %Can do bra-ket notation with \braket{}
\usepackage{framed} %Adds the framed environment
\usepackage{fancyhdr}
\usepackage{datetime} %For formatting of header date
\usepackage{ulem} %Makes strike-through lines with \sout{}
\usepackage{booktabs} %better tables
\usepackage{multirow} %Support multi-row in tables
\usepackage[table,xcdraw]{xcolor} %Support colored rows in tables
\usdate %Month, Dth, YYYY
\geometry{
  letterpaper,
  left=1in,
  right=1in,
  bottom=1in,
  top=1in}
\pagestyle{fancy}
\lhead{NE150 Midterm 1 Study Guide}
\chead{}
\rhead{}
\lfoot{}
\cfoot{\thepage}
\rfoot{\today \quad \currenttime}
\setlength\parindent{0pt}

\begin{document}
\textbf{\Large{Nuclear Engineering 150: Midterm 2 Study Guide}} \\
\vspace{12pt}
%\cite[pp. 45]{krane}
%\cite[Lec 24]{lecture}

\textbf{Disclaimer:} This is not an official study guide. Stuff \sout{might}
\textbf{is} wrong. Use the lecture notes and book!
\vspace{10pt}

\textbf{Note:} Everything in this guide is from the text () or
lecture, or office hours and should be cited as completely as
possible.

\tableofcontents

\section{Neutron Slowing Down}

\subsection{Lethargy}

\begin{itemize}
\item Lethargy ($u$) is a measure of the amount that a neutron has slowed
  down \textit{relative} to an energy $E_0$. It is important to note
  that this is a relative measure, it only tells us about the
  neutron's energy when compared to our reference value.\cite[Lec. 9]{lecture}
  \begin{equation*}
    u = \ln\left(\frac{E_0}{E}\right)=\ln(E_0)-\ln(E)
  \end{equation*}
\item As the neutron slows down, relative to $E_0$, the value of
  lethargy goes \textit{up}. This is why we call it a measure of the
  slowing down. This is why we call it lethargy because it's like a
  measure of the neutron's sleepiness or something; some nuclear
  engineer obviously thought he was being super cute.\cite[Lec. 9]{lecture}
\item Every collision causes a decrease in neutron energy (and
  increase in neutron lethargy). If the neutron goes from $E_i$ to
  $E_f$ after a scattering event, we can solve for the difference in
  the initial and final energies.
  \begin{equation*}
    \Delta{}u=\ln\left(\frac{E_0}{E_f}\right)-\ln\left(\frac{E_0}{E_i}\right)=\ln\left(\frac{E_i}{E_f}\right)
  \end{equation*}
This is also called the logarithmic energy loss.\cite[Lec.9]{lecture}
\end{itemize}

\subsubsection{Average Logarithmic Energy Loss per collision}

\begin{itemize}

\item If we want to know the average logarithmic energy loss per
  collision, we can take the average. We call that squiggle (written
  $\xi$). We will use the energy loss per collision from the last
  midterm, and denote the original energy $E$ and the final energy
  $E'$:
  \begin{equation*}
    \xi = \overline{\ln\left(\frac{E}{E'}\right)} =
    \int^E_{\alpha{}E}dE'\ln\left(\frac{E}{E'}\right)p(E\to{}E')p = \int^E_{\alpha{}E}dE'\ln\left(\frac{E}{E'}\right)\frac{1}{(1-\alpha)E}
  \end{equation*}
Finally, you get:
\begin{equation*}
  \xi = 1+\frac{\alpha}{1-\alpha}\ln(\alpha)
\end{equation*}
Note that this doesn't depend on the original energy $E$. This makes
sense, because it's just an average value, we integrated over all the
possible energies. In the end, it's just a function of what it's
colliding with, because:
\begin{equation*}
  \alpha = \left(\frac{A-1}{A+1}\right)^2
\end{equation*}
\cite[Lec. 9]{lecture}
\item Note that for hydrogen ($A=1$), the value of $\alpha$ is 0. This
  makes $\xi$ undefined ($\ln(0)$ is undefined), so we just set $\xi$
  for hydrogen at 1 (there might be a mathematical way to justify this
  but who cares). This is the highest $\xi$ can be:
  \begin{equation*}
    \xi = 1 = \ln\left(\frac{E}{E'}\right) \to E'=\frac{E}{e}
  \end{equation*}
\item You might be saying ``But study guide, can't the neutron lose
  \textit{all} of its energy in a collision with a proton? Like
  billiard balls?'' and I'd say ``who says billiard balls, they're
  pool balls, get it together.'' Remember this is the \textit{average} energy lost per
  collision so its between the maximum lost (all of it) and the
  minimum lost (none of it).
\item The value
  of $\xi$ gets smaller and smaller with increasing $A$; bouncing a
  ping-pong ball off a basketball won't do much to slow the ping-pong
  ball down.
  \begin{equation*}
    \xi \approx \frac{2}{A+2/3}, \text{for } A>10
  \end{equation*}
Smaller $A \to$ larger $\xi$ (more effective at slowing). \cite[Lec. 9]{lecture}
\item We can use this to figure out the average number of elastic collisions
  required to go from an energy $E_1$ to $E_2$ (Note that we use
  higher numbers for lower energies, this is pretty standard notation
  in neutronics; this is why $E_0$ was our \textit{highest} relative
  energy when defining lethargy).
  \begin{equation*}
    n=\frac{\Delta{}u}{\xi}=\frac{\ln\left(\frac{E_1}{E_2}\right)}{\xi}
  \end{equation*}
  This is just the total logarithmic difference of our two energies,
  divided by the logarithmic amount of energy lost in each collision
  (distance divided by rate).\cite[Lec. 9]{lecture}
\end{itemize}

\subsubsection{$\xi$ for Molecules}
\begin{itemize}
\item If you have a molecule, you sum over the individual values of
  $\xi$, weighted by their cross-sections. This is to take into account
  that there's a different probability that the neutron will scatter
  off of the different elements in the molecule.
  \begin{equation*}
    \bar{\xi}=\frac{1}{\Sigma_s}\sum_i\xi_i\Sigma_{si}
  \end{equation*}
Where $\Sigma$ is the total scattering cross-section for the molecule
and $\Sigma_{Si}$ is the scattering cross-section for each element
$i$.\cite[Lec. 9]{lecture}
\item Remember to get the total scattering cross-section for the
  molecule, you just sum based on how many of each you have. For
  example, for water:
  \begin{equation*}
    \Sigma^{H_2O}_s=N_{H_2O}(2\sigma^H+\sigma^O)
  \end{equation*}
The number densities should all cancel out, letting you just use
microscopic cross-sections.
\end{itemize}

\section{Reactor Criticality}

\subsection{Neutron Population}
\begin{itemize}
\item We generally view the neutron population in our reactor in
  ``generations.'' This is an artificial construct, but is a good way
  to understand the inner mechanics of a reactor. We model our fission
  chain reaction as generating a whole bunch of neutrons at once (a
  generation) that then go on and live their lives. Ultimately, the
  neutrons will either leak, be absorbed, go on to cause more
  fission, etc. The fissions will then create the next generation of
  neutrons.
\item Leakage and absorption are examples of loss mechanisms, and
  fission is a production mechanism. Overall, if we could take the
  whole loss rate $L(t)$ and the whole production rate $P(t)$, we
  could calculate the change in neutrons between generations:
  \begin{equation*}
    \frac{dn(t)}{dt}=P(t)-L(t)
  \end{equation*}
\end{itemize}

\subsubsection{Multiplication Factor}

\begin{itemize}
\item The multiplication factor is the ratio of neutrons in the
  current generation, to those in the last
  generation.\cite[Lec. 10]{lecture}
  \begin{equation*}
    k\equiv\frac{\text{Number of neutrons in this generation}}{\text{Number
        of neutrons in the last generation}}
  \end{equation*}
\item There are three possible situations\cite[Lec. 10]{lecture}:
  \begin{itemize}
  \item[] $k < 1$: Subcritical; there are less neutrons in this
    generation than last, population is decreasing
  \item[] $k=1$: Critical: there are the same number of neutrons in
    this generation as the last, the neutron population is steady.
  \item[] $k>1$: Supercritical: there are more neutrons in this
    generation than the last, population is increasing.
  \end{itemize}
\item Alternatively, we can also define $k$ using the production rate
  of neutrons and the loss rate:
  \begin{equation*}
    k\equiv\frac{P(t)}{L(t)}
  \end{equation*}
You can see how this has the same three situations discussed
above.\cite[Lec. 10]{lecture}
\end{itemize}
\subsubsection{Neutron Population Lifetime}
\begin{itemize}
\item Using this loss rate $L(t)$, we can figure out how long our
  generation would survive (with no production). At a given time $t$,
  we have $n(t)$ neutrons, so using number over loss rate:
  \begin{equation*}
    \ell \equiv \frac{n(t)}{L(t)}
  \end{equation*}
This is our neutron generation lifetime.\cite[Lec 10.]{lecture}
\item We can combine this with our value of $k$ above by examining the
  change in our neutron population over time using the equation from above:
  \begin{equation*}
    \frac{dn(t)}{dt}=P(t)-L(t)=\left[\frac{P(t)}{L(t)}-1\right]L(t)=[k-1]\frac{L(t)}{n(t)}n(t)=\frac{k-1}{\ell}n(t)
  \end{equation*}
We can solve this using an initial condition $n(0)=n_0$:
\begin{equation*}
  n(t)=n_0\text{Exp}\left(\frac{k-1}{\ell}t\right)
\end{equation*}
\end{itemize}

\bibliographystyle{unsrt}
\bibliography{../NE150}
\end{document} 