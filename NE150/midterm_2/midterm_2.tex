\documentclass[letter]{article}

\usepackage{amsmath}
\usepackage{graphicx}
\usepackage{geometry}
\usepackage{braket} %Can do bra-ket notation with \braket{}
\usepackage{framed} %Adds the framed environment
\usepackage{fancyhdr}
\usepackage{datetime} %For formatting of header date
\usepackage{ulem} %Makes strike-through lines with \sout{}
\usepackage{booktabs} %better tables
\usepackage{multirow} %Support multi-row in tables
\usepackage[table,xcdraw]{xcolor} %Support colored rows in tables
\usdate %Month, Dth, YYYY
\geometry{
  letterpaper,
  left=1in,
  right=1in,
  bottom=1in,
  top=1in}
\pagestyle{fancy}
\lhead{NE150 Midterm 1 Study Guide}
\chead{}
\rhead{}
\lfoot{}
\cfoot{\thepage}
\rfoot{\today \quad \currenttime}
\setlength\parindent{0pt}

\begin{document}
\textbf{\Large{Nuclear Engineering 150: Midterm 2 Study Guide}} \\
\vspace{12pt}
%\cite[pp. 45]{krane}
%\cite[Lec 24]{lecture}

\textbf{Disclaimer:} This is not an official study guide. Stuff \sout{might}
\textbf{is} wrong. Use the lecture notes and book!
\vspace{10pt}

\textbf{Note:} Everything in this guide is from the text () or
lecture, or office hours and should be cited as completely as
possible.

\tableofcontents

\section{Neutron Slowing Down}

\subsection{Lethargy}

\begin{itemize}
\item Lethargy ($u$) is a measure of the amount that a neutron has slowed
  down \textit{relative} to an energy $E_0$. It is important to note
  that this is a relative measure, it only tells us about the
  neutron's energy when compared to our reference value.\cite[Lec. 9]{lecture}
  \begin{equation*}
    u = \ln\left(\frac{E_0}{E}\right)=\ln(E_0)-\ln(E)
  \end{equation*}
\item As the neutron slows down, relative to $E_0$, the value of
  lethargy goes \textit{up}. This is why we call it a measure of the
  slowing down. This is why we call it lethargy because it's like a
  measure of the neutron's sleepiness or something; some nuclear
  engineer obviously thought he was being super cute.\cite[Lec. 9]{lecture}
\item Every collision causes a decrease in neutron energy (and
  increase in neutron lethargy). If the neutron goes from $E_i$ to
  $E_f$ after a scattering event, we can solve for the difference in
  the initial and final energies.
  \begin{equation*}
    \Delta{}u=\ln\left(\frac{E_0}{E_f}\right)-\ln\left(\frac{E_0}{E_i}\right)=\ln\left(\frac{E_i}{E_f}\right)
  \end{equation*}
This is also called the logarithmic energy loss.\cite[Lec.9]{lecture}
\item If we want to know the average logarithmic energy loss per
  collision, we can take the average. We call that squiggle (written
  $\xi$). We will use the energy loss per collision from the last
  midterm, and denote the original energy $E$ and the final energy
  $E'$:
  \begin{equation*}
    \xi = \overline{\ln\left(\frac{E}{E'}\right)} =
    \int^E_{\alpha{}E}dE'\ln\left(\frac{E}{E'}\right)p(E\to{}E')p = \int^E_{\alpha{}E}dE'\ln\left(\frac{E}{E'}\right)\frac{1}{(1-\alpha)E}
  \end{equation*}
Finally, you get:
\begin{equation*}
  \xi = 1+\frac{\alpha}{1-\alpha}\ln(\alpha)
\end{equation*}
Note that this doesn't depend on the original energy $E$. This makes
sense, because it's just an average value, we integrated over all the
possible energies. In the end, it's just a function of what it's
colliding with, because:
\begin{equation*}
  \alpha = \left(\frac{A-1}{A+1}\right)^2
\end{equation*}
\cite[Lec. 9]{lecture}
\end{itemize}


\bibliographystyle{unsrt}
\bibliography{../NE150}
\end{document}